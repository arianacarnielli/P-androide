%%%%%%%%%%%%%%%%%%%%%%%%%%%%%%%%%%%%%%%%%%%%%%%%%%%%%%%%%%%%%%%%%%%%%%%%%%%%%%%%%%%%%
% PACOTES                                                                           %
%%%%%%%%%%%%%%%%%%%%%%%%%%%%%%%%%%%%%%%%%%%%%%%%%%%%%%%%%%%%%%%%%%%%%%%%%%%%%%%%%%%%%
\documentclass[a4paper,11pt]{article}

%-----------------------------------------------------------------------------------%
% LAYOUT DA PÁGINA                                                                  %
%-----------------------------------------------------------------------------------%
\usepackage[top=2.25cm, bottom=2.25cm, left=2.25cm, right=2.25cm]{geometry}
%\usepackage{fancyhdr} % Permite controlar como são exibidos os cabeçalhos

%-----------------------------------------------------------------------------------%
% FORMATAÇÃO DO TEXTO                                                               %
%-----------------------------------------------------------------------------------%
%\usepackage{setspace} % Permite definir o espaçamento entre linhas

%-----------------------------------------------------------------------------------%
% PACOTES DE IMAGENS                                                                %
%-----------------------------------------------------------------------------------%
\usepackage[pdftex]{graphicx}
\pdfsuppresswarningpagegroup=1 % A warning issued when several PDF images are
% imported in the same page. Mostly harmless, can be almost always supressed.
%\usepackage[pstarrows]{pict2e} % Amplia as funcionalidades do ambiente picture
\usepackage{tikz}
\usetikzlibrary{shapes, arrows, arrows.meta}

%-----------------------------------------------------------------------------------%
% PACOTES DE TABELAS                                                                %
%-----------------------------------------------------------------------------------%
\usepackage{array} % Facilita a formatação de tabelas
%\usepackage{multirow} % Permite criar células que ocupam várias linhas em uma tabela
\usepackage{longtable} % Permite criar tabelas que quebram de página

%-----------------------------------------------------------------------------------%
% PACOTES MATEMÁTICOS DE BASE                                                       %
%-----------------------------------------------------------------------------------%
\usepackage{amsfonts,amstext,amscd,bezier,amsthm,amssymb}
\usepackage[centertags]{amsmath}

%-----------------------------------------------------------------------------------%
% PACOTES DE SÍMBOLOS MATEMÁTICOS                                                   %
%-----------------------------------------------------------------------------------%
\usepackage{mathtools} % Símbolos matemáticos extras. (ex.: \xrightharpoon)
%\usepackage[integrals]{wasysym} % Muda o estilo das integrais, além de outros
%                                 símbolos extras
%\usepackage[nice]{nicefrac} % Permite o uso de frações "melhores". Usar \nicefrac{}{}

%-----------------------------------------------------------------------------------%
% PACOTES DE FONTES MATEMÁTICAS                                                     %
%-----------------------------------------------------------------------------------%
%\usepackage{mathbbol} % Quase todos os símbolos com \mathbb
%\usepackage{bbm} % Extensão dos símbolos de \mathbb. Usar comando \mathbbm
%\usepackage{calrsfs} % Muda o estilo de \mathcal
%\usepackage[mathcal]{euscript} % Muda o estilo de \mathcal

%-----------------------------------------------------------------------------------%
% PACOTES DE CODIFICAÇÃO DE FONTES                                                  %
%-----------------------------------------------------------------------------------%
\usepackage[utf8]{inputenc} % Permite o uso de caracteres ISO 8859-1, incluindo os
%                               caracteres acentuados diretamente.
\usepackage[T1]{fontenc} % Uso de fontes T1, necessário para tratar caracteres
%                          acentuados como um único bloco.

%-----------------------------------------------------------------------------------%
% PACOTES DE LÍNGUAS                                                                %
%-----------------------------------------------------------------------------------%
\usepackage[french]{babel} % Seleciona a língua do documento, definindo nomes de
%                              seções, nome do índice, da bibliografia, etc. Em caso
%                              de documento com mais de uma língua, a padrão é a
%                              última.
\NoAutoSpaceBeforeFDP % Utilizar em francês se quiser evitar espaços antes de :

%-----------------------------------------------------------------------------------%
% PACOTES DE BIBLIOGRAFIA                                                           %
%-----------------------------------------------------------------------------------%
%\usepackage{babelbib} % Permite definir a língua das entradas da bibliografia. Usar
%                       [fixlanguage] para uma mesma língua para todas as entradas e
%                       \selectbiblanguage{} para definir a língua. Um estilo compa-
%                       tível com babelbib deve ser usado (ex: babplain)
\usepackage{cite} % Organiza os elementos citados dentro de um mesmo \cite.

%-----------------------------------------------------------------------------------%
% PACOTES DE FONTES                                                                 %
%-----------------------------------------------------------------------------------%
% Computer Modern (fonte padrão)                                                    %
% - - - - - - - - - - - - - - - - - - - - - - - - - - - - - - - - - - - - - - - - - %
%\usepackage{ae} % A usar com a fonte padrão do LaTeX quando forem gerados PDFs, para
%                 corrigir erros de visualização

% Computer Modern Bright (sans serif)                                               %
% - - - - - - - - - - - - - - - - - - - - - - - - - - - - - - - - - - - - - - - - - %
%\usepackage{cmbright}

% Times New Roman                                                                   %
% - - - - - - - - - - - - - - - - - - - - - - - - - - - - - - - - - - - - - - - - - %
%\usepackage{mathptmx} % Muda texto e modo matemático
%\usepackage{times} % Apenas texto, não muda modo matemático

% Arial                                                                             %
% - - - - - - - - - - - - - - - - - - - - - - - - - - - - - - - - - - - - - - - - - %
%\usepackage[scaled]{uarial} % Arial como fonte sans serif padrão

% Palatino                                                                          %
% - - - - - - - - - - - - - - - - - - - - - - - - - - - - - - - - - - - - - - - - - %
%\usepackage{mathpazo} % Muda texto e modo matemático
%\usepackage{palatino} % Apenas texto, não muda modo matemático

% Concrete                                                                          %
% - - - - - - - - - - - - - - - - - - - - - - - - - - - - - - - - - - - - - - - - - %
%\usepackage{ccfonts} % Texto: Concrete; Matemático: Concrete Math
%\usepackage{ccfonts, eulervm} % Texto: Concrete; Matemático: Euler

% Iwona                                                                             %
% - - - - - - - - - - - - - - - - - - - - - - - - - - - - - - - - - - - - - - - - - %
%\usepackage[math]{iwona} % Texto e modo matemático: Iwona

% Kurier                                                                            %
% - - - - - - - - - - - - - - - - - - - - - - - - - - - - - - - - - - - - - - - - - %
%\usepackage[math]{kurier} % Texto e modo matemático: Kurier

% Antykwa Póltawskiego                                                              %
% - - - - - - - - - - - - - - - - - - - - - - - - - - - - - - - - - - - - - - - - - %
%\usepackage{antpolt} % Texto: Antykwa Póltawskiego; Matemático: nenhum
                     % Usar fontenc = QX ou OT4

% Utopia                                                                            %
% - - - - - - - - - - - - - - - - - - - - - - - - - - - - - - - - - - - - - - - - - %                     
%\usepackage{fourier} % Texto: Utopia; Matemático: Fourier

% KP Serif                                                                          %
% - - - - - - - - - - - - - - - - - - - - - - - - - - - - - - - - - - - - - - - - - %
\usepackage{kpfonts}

%-----------------------------------------------------------------------------------%
% CORES                                                                             %
%-----------------------------------------------------------------------------------%
\usepackage{color}
\definecolor{darkgreen}{rgb}{0,0.5,0}
\definecolor{darkmagenta}{rgb}{0.5,0,0.5}
\definecolor{darkgray}{rgb}{0.5,0.5,0.5}
\definecolor{darkblue}{rgb}{0.2,0.2,0.4}
\definecolor{darkred}{rgb}{0.6,0.15,0.15}
\definecolor{gray}{rgb}{0.65,0.65,0.65}
\definecolor{lightgray}{rgb}{0.8,0.8,0.8}
\definecolor{lightblue}{rgb}{0.5,0.5,1}
\definecolor{lightgreen}{rgb}{0.5,1,0.5}
\definecolor{deadred}{rgb}{0.7, 0.2, 0.2}
\definecolor{deadblue}{rgb}{0.2, 0.2, 0.7}

%-----------------------------------------------------------------------------------%
% PACOTES DIVERSOS                                                                  %
%-----------------------------------------------------------------------------------%
\usepackage{icomma} % Permite uso de vírgula como separador decimal
\usepackage{url} % Pacote para não ter problemas com URLs. Usar \url{}
%\usepackage{randtext} % Troca a ordem de letras de uma frase (útil com e-mails em
                      % PDFs a serem publicados on-line.
\usepackage[hidelinks]{hyperref}
%\usepackage{showkeys} % Para mostrar o nome dos labels
\usepackage{enumitem} % Facilita o uso de listas, inclusive referências a itens de
                      % listas.
%\usepackage[absolute]{textpos} % Posição absoluta de texto na página
%\usepackage{pdfpages} % Permite incluir documentos em PDF no arquivo
%\usepackage{refcheck} % Verifica as referências procurando por
%                      % labels não usados ou equações numeradas sem labels.
%                      % Verificar o arquivo .log e procurar por RefCheck.
\usepackage[french, onelanguage]{algorithm2e}

%%%%%%%%%%%%%%%%%%%%%%%%%%%%%%%%%%%%%%%%%%%%%%%%%%%%%%%%%%%%%%%%%%%%%%%%%%%%%%%%%%%%%
% CONFIGURAÇÕES                                                                     %
%%%%%%%%%%%%%%%%%%%%%%%%%%%%%%%%%%%%%%%%%%%%%%%%%%%%%%%%%%%%%%%%%%%%%%%%%%%%%%%%%%%%%

%-----------------------------------------------------------------------------------%
% FORMATAÇÃO DO TEXTO                                                               %
%-----------------------------------------------------------------------------------%
%\onehalfspacing % Espaçamento 1 1/2 (definido no pacote setspace)

%-----------------------------------------------------------------------------------%
% DEFINIÇÃO DE AMBIENTES MATEMÁTICOS                                                %
%-----------------------------------------------------------------------------------%
\theoremstyle{plain}
\newtheorem*{prop}{Proposition}

%-----------------------------------------------------------------------------------%
% DEFINIÇÃO DE COMANDOS MATEMÁTICOS                                                 %
%-----------------------------------------------------------------------------------%
%\newcommand*\diff{\mathop{}\!\mathrm{d}}

%\newcommand{\norm}[1]{\left\lVert #1\right\lVert} % Norma
%\newcommand{\abs}[1]{\left\lvert #1\right \rvert} % Valor absoluto
%\newcommand{\floor}[1]{\left\lfloor #1 \right\rfloor} % Arredondar para baixo
%\newcommand{\ceil}[1]{\left\lceil #1 \right\rceil} % Arredondar para cima
\DeclarePairedDelimiter{\ceil}{\lceil}{\rceil}
\DeclarePairedDelimiter{\floor}{\lfloor}{\rfloor}
\DeclarePairedDelimiter{\abs}{\lvert}{\rvert}
\DeclareMathOperator*{\argmax}{argmax}

%-----------------------------------------------------------------------------------%
% NUMERAÇÃO DE ELEMENTOS                                                            %
%-----------------------------------------------------------------------------------%
%\numberwithin{table}{section}
%\numberwithin{table}{subsection}
%\numberwithin{figure}{section}
%\numberwithin{figure}{subsection}
%\numberwithin{equation}{section}
%\numberwithin{equation}{subsection}
%\numberwithin{theo}{chapter}
%\numberwithin{theo}{subsection}

% Maximal percentage of the page occupied by floats
\renewcommand\floatpagefraction{.9}
\renewcommand\topfraction{.9}
\renewcommand\bottomfraction{.9}
\renewcommand\textfraction{.1}
% Maximal number of floats per page
\setcounter{totalnumber}{50}
\setcounter{topnumber}{50}
\setcounter{bottomnumber}{50}

%%%%%%%%%%%%%%%%%%%%%%%%%%%%%%%%%%%%%%%%%%%%%%%%%%%%%%%%%%%%%%%%%%%%%%%%%%%%%%%%%%%%%
% ESTRUTURA DO DOCUMENTO                                                            %
%%%%%%%%%%%%%%%%%%%%%%%%%%%%%%%%%%%%%%%%%%%%%%%%%%%%%%%%%%%%%%%%%%%%%%%%%%%%%%%%%%%%%
\begin{document}

\setlist[itemize]{label=\textbullet, nosep}

%\vspace*{-2.5em}

\pagestyle{plain}

%\title{Projet UE COMPLEX --- MU4IN900 \\ Couverture de graphe}
%\author{Ariana Carnielli}
%\date{}
%\maketitle


%\tableofcontents

\section{État de l'art}

\subsection{Description générale du problème de \emph{Troubleshooting}}

Le projet se concentre sur le problème de \emph{Troubleshooting}, c'est-à-dire sur le problème suivant~: étant donné un dispositif en panne, on cherche une stratégie de réparation de coût total minimal. Plus précisément, on considère que le dispositif est constitué d'un nombre fini de composantes $c_1, \dotsc, c_n$, certaines pouvant être en panne, et que l'on dispose de 2 types d'actions qui peuvent être réalisées de façon séquentielle~: observations et réparations. 

Les observations, que l'on dénote par $o_1, \dotsc, o_m$,  peuvent être ``locales'', c'est-à-dire d'une seule composante du dispositif, ou ``globales'' quand elles dépendent de plusieurs composantes. Il peut y avoir de composantes qui n'ont pas d'observation locale associée. On considère aussi qu'on a une observation spéciale $o_0$ qui porte sur l'état général du dispositif. Les réparations portent toujours sur une seule composante à la fois et on note $r_i$ la réparation de la composante $i$. 

Les ensembles d'observations, réparations et actions sont dénotés respectivement par $\mathcal O = \{ o_0, \dotsc, o_m\}$, $\mathcal R = \{r_1, \dotsc, r_n\}$ et $\mathcal A = \mathcal O \cup \mathcal R$. Chaque action $a \in \mathcal A$ a un coût associé $C(a) \geq 0$ et l'objectif est donc de mettre le dispositif en état de marche en minimisant le coût total \[\sum_i C(a_i),\] où $a_i$ appartiennent à l'ensemble des actions prises. Dans certaines situations, il peut être intéressant de rajouter à $\mathcal A$ une action spéciale, $a_0$, correspondant à ``appeler le service'', qui résout le problème avec certitude mais a un coût très élevé, représentant, par exemple, la possibilité d'envoyer le dispositif à un centre plus spécialisé ou d'en acheter un nouveau.

\subsection{Théorie de la décision et fonctions d'utilité}

\subsubsection{Théorie classique}

Le problème de minimisation présenté comporte des difficultés liées aux incertitudes: on ne connaît pas quelles sont les composantes défectueuses, quels seront les résultats des observations ni les conséquences d'une réparation sur l'ensemble du dispositif. Il nous faut ainsi prend des décisions dans l'incertain et, afin de traiter ce problème, on se sert des outils de la théorie de la décision telle que décrite de façon générale dans \cite{North_1968}. Il s'agit d'un cadre formel qui permet de faire des choix d'actions parmi des alternatives lorsque les conséquences de ces actions ne sont connues que dans un sens probabiliste. Cette théorie repose sur la modélisation probabiliste et la représentation des préférences d'un agent, de façon synthétique, à travers une fonction d'utilité. 

L'idée d'une fonction d'utilité est de donner des valeurs numériques à des résultats. Une hypothèse fondamentale faite pour cela est de supposer que chaque paire de résultats peut être comparée : un des résultats sera forcément meilleur ou aussi bon que l'autre. On demande aussi à ce que cette comparaison des résultats soit transitive : préférer $A$ à $B$ et $B$ à $C$ implique préférer $A$ à $C$.
Une autre hypothèse fondamentale est que l'on peut comparer non seulement des résultats purs mais aussi des loteries, c'est-à-dire des situations où l'on peut avoir quelques résultats avec une certaine probabilité. En particulier, si un agent préfère $A$ à $B$, alors, si on lui donne le choix entre deux loteries entre $A$ et $B$, l'agent préférera la loterie donnant plus de probabilité à $A$. 

Les loteries n'ont pas de valeur intrinsèque, ce qui implique que l'on n'a pas intérêt à faire des loteries imbriquées, où le prix d'une loterie serait de participer à une deuxième loterie. Finalement, on suppose une continuité des loteries: si l'agent a l'ordre de préférence $A > C > B$, alors il existe une loterie entre $A$ et $B$ telle que l'agent sera indifférent entre cette loterie et le résultat $C$. Sous ces hypothèses, il est possible de condenser les préférences de l'agent dans une fonction d'utilité $u$ telle que $u(A) > u(B)$ si et seulement si l'agent préfère $A$ à $B$. En plus, si l'agent est indifférent entre $C$ et une loterie entre $A$ et $B$ avec probabilités respectives $P$ et $1-P$, alors \[u(C) = Pu(A) + (1-P)u(B),\] c'est-à-dire, l'utilité de la loterie est l'espérance de son gain.

Dans un cadre simple avec une seule action à prendre parmi $\alpha_1, \dotsc, \alpha_k$ et des résultats possibles entre $R_1, \dotsc, R_\ell$, on calcule l'espérance de l'utilité de l'action $\alpha_i$ par \[Eu(\alpha_i) = \sum_{j = 1}^\ell u(R_j)P(R_j\mid\alpha_i)\] et on choisit celle qui maximise cette utilité espérée. En général, on doit faire face à des problèmes avec une séquence de décisions que l'on peut représenter par un arbre, comme celui de la Figure \ref{arbre}, mais dont le calcul exhaustif en général est trop complexe, nécessitant ainsi de méthodes d'approximation permettant de résoudre le problème en temps raisonnable.

\begin{figure}[ht]
\centering
\begin{tikzpicture}
\node[draw, rectangle] (n1) at (0, 0) {};
\node[draw, circle] (n2) at (2, 2) {};
\node[draw,circle] (n3) at (2, -2) {};
\node (n4) at (4, 3) {$R_1$};
\node[draw, rectangle] (n5) at (4, 1) {};
\node (n6) at (4, -1) {$R_2$};
\node (n7) at (4, -3) {$R_3$};
\node[draw, circle] (n8) at (6, 2) {};
\node[draw, circle] (n9) at (6, 0) {};
\node (n10) at (8, 2) {$R_4$};
\node (n11) at (8, 0) {$R_2$};
\node (n12) at (8, -1) {$R_4$};

\draw (n1) -- node[midway, above left] {$\alpha_1$} (n2) -- (n4);
\draw (n2) -- (n5) -- node[midway, above left] {$\alpha_3$} (n8) -- (n10);
\draw (n5) -- node[midway, below left] {$\alpha_4$} (n9) -- (n11);
\draw (n9) -- (n12);
\draw (n1) -- node[midway, below left] {$\alpha_2$} (n3) -- (n6);
\draw (n3) -- (n7);
\end{tikzpicture}
\caption{Arbre avec une séquence d'au plus deux décisions à prendre. Les nœuds carrés représentent les décisions à prendre et ceux en forme de cercle, les loteries.}
\label{arbre}
\end{figure}

Dans le cadre du \emph{Troubleshooting}, les actions $\alpha_i$ sont des observations ou réparations de $\mathcal A$. L'utilité d'un résultat $R_i$ est le coût des actions qui mènent de la racine à la feuille $R_i$, la maximisation de l'utilité est remplacée par la minimisation du coût, et tous les résultats $R_i$ représentent la même situation, à savoir le fonctionnement normal du dispositif.

\subsubsection{Ensemble optimal de recommandations}
Un autre problème intéressant de la théorie de décision c’est de déterminer un ensemble des alternatives plus ou moins ``convenables'' parmi celles proposées. Plus particulièrement, étant donné un décideur avec sa propre utilité l’on doit trouver \emph{m} les meilleures solutions parmi $\alpha_1$, $\alpha_2$, $\dotsc$, $\alpha_n$ (\emph{m} < \emph{n}). Une approche la plus simple consiste à trier les stratégies au-dessus à l’ordre de décroissance de leurs utilités espérées, ensuite, l’on prend \emph{m} premières décisions de la liste triée. En pratique, cette approche n’est pas toujours suffisante \cite{price_optimal_2005} car dans la grande majorité de cas on ne connaît l’utilité d’acteur qu’avec l’incertain et, alors, ce serait possible de choisir \emph{m} solutions assez similaire qui ne marcherait pas, en particulier, si \emph{n} est relativement grand. Par exemple, supposons que l’on gère un magasin de livres en ligne et l’on cherche un ensemble de 10 livres à montrer à l’utilisateur particulier pour lui intéresser le plus que possible. Soit l’on a élicité son utilité et l’on lui propose les 10 meilleures alternatives selon cette utilité. Comme chaque livre a presque toujours des éditions différentes, l’on suppose que c’est bien possible de ne montrer à cet utilisateur qu’une seule proposition des parutions diverses toutes les 10 fois. Bien que l’on puisse introduire un critère de similarité selon lequel l’on filtre des alternatives, une approche différente a été proposée dans l’article \cite{price_optimal_2005} : l’on définit plutôt un critère pour chaque sous-ensemble de $\{\alpha_1, \alpha_2, \dotsc, \alpha_n\}$ et, après, l’on cherche telle sous-ensemble qui maximisera ce critère. L’on appelle ce sous-ensemble comme \emph{ensemble optimal de recommandations} (de \emph{Optimal Recommendation Set} en anglais). Il nous faudrait noter que cette technique assure une \emph{diversité} des alternatives dans l’ensemble trouvé ce qui nous permettrait de proposer à l’utilisateur des bonnes solutions malgré l’incertain de son utilité. En outre, l’article concerné fournit une recherche sur la complexité du problème ainsi que certains algorithmes pour le résoudre.

\subsection{Réseaux bayésiens}

Les incertitudes dans le problème de \emph{Troubleshooting} ont plusieurs origines~: on ne connaît pas quelles composantes sont en panne ni les effets précis que le changement de l'état d'une composante peut avoir sur les autres autres composantes, sur les observations et sur l'état du système. Ces incertitudes sont représentées dans notre approche par des probabilités. La représentation intégrale de la loi de probabilité jointe de toutes les composantes et de toutes les observations du système serait trop gourmande en mémoire et inutilement complexe puisque l'on peut imaginer qu'il y a plusieurs relations d'indépendance ou d'indépendance conditionnelle entre elles. Ainsi, les réseaux bayésiens, décrits par exemple dans \cite{Jensen_2007}, sont un outil mathématique adaptée à la représentation des probabilités de notre problème.

\subsection{Approches au problème de \emph{Troubleshooting}}

\subsubsection{Approche exacte}

L'approche immédiate pour résoudre le problème \emph{Troubleshooting} est de construire l'arbre correspondant avec toutes les actions d'observation et réparation qui peuvent être réalisées à chaque étape, les feuilles correspondant aux états où le dispositif a été réparé après une séquence d'actions. Cette approche permet de résoudre le problème de façon exacte, mais en temps exponentiel en la taille des données car il faut parcourir toute l'arbre pour déterminer le meilleur choix en chaque nœud.

Le fait que cet algorithme exacte est exponentiel n'est pas surprenant : il a été démontré dans \cite{Vomlelov__2003} que, sauf sous des hypothèses simplificatrices assez fortes (par exemple, absence d'observations et présence d'une unique composante panne sur le système), le problème de \emph{Troubleshooting} est NP-difficile. Cela motive ainsi la recherche d'heuristiques donnant de bonnes solutions pratiques ainsi que d'algorithmes approchés pour le problème de \emph{Troubleshooting}. Il faut cependant noter que, en toute généralité, le problème de \emph{Troubleshooting} est aussi NP-difficile à résoudre dans un sens approché, comme démontré dans \cite{L_n_2014}.

\subsubsection{Algorithme exacte sous hypothèses simplificatrices}

Sous des hypothèses simplificatrices assez restrictives, il est connu \cite{heckerman1994troubleshooting, Heckerman_1995, Vomlelov__2003, L_n_2014} qu'il est possible de résoudre le problème de \emph{Troubleshooting} en temps polynomial. Plus précisément, on suppose que~:
\begin{itemize}
\item il n'y a qu'une seule composante présentant un défaut~;
\item les coûts des réparations sont indépendants~; 
\item la seule observation possible est celle de l'état du dispositif, $o_0$, qui a un coût $0$.
\end{itemize}
À travers le réseau bayésien décrivant le dispositif, on dispose, pour tout $i \in \llbracket 1, n\rrbracket$, de la probabilité $p_i$ que la réparation $r_i$ résout le problème. L'algorithme polynomial consiste alors à calculer les rapports $\frac{p_i}{C(r_i)}$ et réparer les composantes dans l'ordre décroissant de ces rapports, observant après chaque réparation si le dispositif marche ou pas.

%Les problèmes de \emph{Troubleshooting} ont été efficacement modélisés de façon approchée en utilisant des réseaux Bayésiens dans \cite{heckerman1994troubleshooting, Heckerman_1995}. Une première approche utilise des hypothèses assez restrictives et donne un algorithme final très efficace. 

Ces hypothèses sont trop restrictives car, d'une part, il n'est pas réaliste de supposer qu'on n'a qu'une seule composante en panne et, d'autre part, on dispose souvent d'autres observations outre $o_0$ et les informations qu'elles peuvent apporter peuvent être assez importantes pour que l'on les ignore. Cette deuxième remarque est en lien avec la notion de \emph{valeur de l'information}~: la valeur qu'une information apporte, dans le cadre du \emph{Troubleshooting}, est la différence entre le coût espéré de réparation sans cette information et le coût en prenant en compte l'information (auquel on ajoute aussi le coût d'obtention de l'information à travers une observation). Cette notion s'applique à des problèmes de décision plus généraux que le \emph{Troubleshooting}, comme décrit dans \cite{Braziunas_2008} pour les problèmes d'élicitation décrit plus en détail ci-après.

\subsubsection{Paires observation-réparation}

Les articles \cite{heckerman1994troubleshooting, Heckerman_1995} présentent un algorithme heuristique pour le problème de \emph{Troubleshooting} qui fait des hypothèses moins restrictives que les précédentes. On suppose désormais qu'il peut y avoir plusieurs composantes en panne mais on restreint les observations que l'on peut faire. Pour les composantes non-observables (c'est-à-dire, qui n'ont pas d'observation locale associée), la seule action disponible est leur réparation. Pour les composantes observables, on impose de toujours faire l'observation locale correspondante avant la réparation, ce qui est appelé une paire observation-réparation. Ainsi, on restreint l'ensemble $\mathcal A$ des actions possibles aux réparations de composantes non-observables et aux paires observation-réparation pour les autres composantes. Le coût de la paire $(o_i, r_j)$ est
\[C(o_i) + P(o_i \neq \text{normal} \mid E) C(r_j),\]
où $E$ représente les informations dont on dispose. L'algorithme suit la même idée que le précédent, en choisissant à chaque étape le plus grand rapport probabilité/coût, mais ces rapports doivent désormais être recalculés à chaque étape car les probabilités et les coûts évoluent en fonction des actions déjà effectuées.

\subsubsection{Approche myope}

La contribution principale de \cite{heckerman1994troubleshooting, Heckerman_1995} est une autre approche heuristique qui, par rapport au cas précédent, rajoute la possibilité de faire des observations globales en dehors des paires ob\-ser\-va\-tion-ré\-pa\-ra\-tion. Pour éviter la complexité du cas général, ils développent une technique appelée \emph{myope}, qui consiste à calculer l'espérance de coût après une observation globale $o_i$ de façon approchée en supposant qu'aucune autre observation globale ne sera faite dans la suite. À chaque étape, on compare ces espérances de coût (une pour chaque observation globale) avec l'espérance de coût sans observation (celle que l'on obtiendrait en appliquant l'algorithme précédent) pour décider s'il est intéressant de réaliser une observation globale à cette étape ou pas. Cela revient aussi à déterminer si la valeur de l'information apportée par l'observation est positive ou pas.

Les travaux \cite{heckerman1994troubleshooting, Heckerman_1995} présentent aussi des méthodes pour calculer les probabilités de réparation en utilisant des réseaux Bayésiens. Pour ce faire, il est nécessaire non seulement de calculer ces probabilités mais aussi de les mettre à jour en fonction des informations acquises lors d'observations et de réparations. Afin de simplifier ce calcul, les articles introduisent la notion de réseaux de réponse, construits à partir d'un réseau Bayésien et d'une action effectuée. Des simplifications supplémentaires sont encore possibles sous l'hypothèse d'indépendance causale. Cet algorithme a été testé et validé pour certains modèles concrets.

\subsubsection{Extensions de l'approche myope}

Des extensions de l'approche de \cite{heckerman1994troubleshooting, Heckerman_1995} ont été proposées en particulier dans \cite{Jensen_2001, Langseth_2003} où les méthodes développés ont permis d'obtenir des résultats assez efficaces pour des cas plus généraux. Plus spécifiquement, l’article \cite{Jensen_2001} considère, d’abord, que les composantes et les actions de réparation ne sont plus en correspondance univoque et que chaque action peut traiter les composantes associées avec une certaine probabilité. Ainsi, les actions ne sont plus parfaites et ne conduisent pas toujours vers une réparation d’une composante. Par ailleurs, on suppose que chaque action a la possibilité de dépanner plusieurs composantes et, réciproquement, chaque composante peut être réparée par des actions différentes. De plus, cet article propose une approche qui améliore la technique myope décrite ci-dessus. %L’on discutera comment ces modifications influencent un rapport d’approximation du problème ci-dessous. 

Quant à l’article \cite{Langseth_2003}, l'auteur y considère un cas encore plus général où chaque composante est constituée de sous-composantes qui peuvent elles-mêmes être à l'origine de la panne du dispositif et qui peuvent être réparées. En outre, les composantes, vues comme des ensembles de sous-composantes, ne sont pas forcément deux-à-deux disjointes. En conséquence, dans ce modèle il est possible qu’une sous-composante $X$ soit partie de deux composantes différentes, $c_i$ et $c_j$, $i \neq j$. Selon des résultats de simulations numériques de \cite{Langseth_2003}, les algorithmes y développés retournent des stratégies pour le \emph{troubleshooting} beaucoup plus efficaces que l'approche myope pour des problèmes concrets. En effet, pour les modèles considérés, les techniques de \cite{Jensen_2001, Langseth_2003} retournent des solutions dont le coût espéré de réparation a un écart relatif moyen de 2,51\% par rapport à l’optimum trouvé par une recherche exhaustive, au lieu de 21,5\% pour l’approche myope.

\subsection{Élicitation}

\subsubsection{Élicitation dans le \emph{Troubleshooting}}

Dans les problèmes de la théorie de la décision en général, la fonction d'utilité n'est pas parfaitement connue et il nous faut alors des mécanismes d'élicitation permettant de l'estimer. Cela est bien le cas du problème de \emph{Troubleshooting} qui nous intéresse~: malgré le fait que l'utilité provient du coût et que les coûts des actions individuelles sont connus, la connaissance du coût de réparation de façon exacte impliquerait un parcours complet de l'arbre des décisions, ce qui n'est pas réalisable dans la plupart des cas. L'article \cite{Braziunas_2008} présente le problème d'élicitation des fonctions d'utilité et donne un panorama des techniques pour le résoudre.

L'idée principale est de considérer que la fonction d'utilité dépend des résultats à travers d'un certain nombre de caractéristiques de ces résultats. Autrement dit, on représente un résultat $R$ comme un vecteur de caractéristiques $(x_1, \dotsc, x_d)$ et on regarde $u(R)$ comme $u(x_1, \dotsc, x_d)$. L'article \cite{Braziunas_2008} suppose alors que $u$ satisfait une hypothèse d'\emph{indépendance additive généralisée}, ce qui permet de l'écrire comme combinaison linéaire de fonctions $u_1, \dotsc, u_p$, chacune dépendant seulement d'une partie des caractéristiques $x_i$, par exemple
\[u(x_1, x_2, x_3, x_4) = \lambda_1 u_1(x_1) + \lambda_2 u_2(x_2, x_4) + \lambda_3 u_3(x_3, x_4).\]
Ainsi, l'élicitation peut être décomposée en deux étapes, une locale correspondant à estimer les $u_i$ et une globale afin de déterminer les $\lambda_i$. Il présente aussi deux techniques pour représenter les incertitudes sur la fonction d'utilité, basées sur une approche Bayésienne et une approche ensembliste. Celle qui nous intéresse est la Bayésienne, qui repose sur la notion de valeur de l'information, comme expliqué précédemment.

\subsubsection{Valeur de l'information}

Les concepts de bases de la théorie de la valeur de l’information sont présentés dans l’article \cite{howard_information_1966}. Plus spécifiquement, supposons que l’on doit prendre une décision \emph{D} dans l’incertain parmi celles proposées sachant quand même quelque évidence $\varepsilon$. Soit, en plus, il existe un voyant qui est capable de nous partager une autre évidence \emph{X} en diminuant de telle manière l’incertain. Néanmoins, ce ne serait pas gratuit : pour qu’un voyant admette à nous dire son secret il faut lui payer $\emph{C}_\emph{X}$. Serait-ce alors mieux de payer au voyant ou l’on doit prendre plutôt une décision avec aucunes d’autre informations ? L’idée principale de l’approche y proposée c’est de comparer une utilité que l’on pourrait obtenir sans et avec l’évidence fournie par voyant. Soit $\emph{Eu}(\emph{D} \mid \varepsilon)$ est notre utilité espérée si l’on prend une décision \emph{D} sachant $\varepsilon$. Selon une technique proposée, il suffit d’observer une valeur
\[\emph{Eu}_{\emph{X}} =\max_{\emph{D}} \emph{Eu}(\emph{D} \mid \emph{X}, \varepsilon) - \max_{\emph{D}} \emph{Eu}(\emph{D} \mid \varepsilon),\]
donc la différence entre l’utilité espérée maximale si l’on connaît \emph{X} et celle sinon. Selon ce critère, l’on décide alors de payer pour évidence \emph{X} si $\emph{Eu}_\emph{X} > 0$ et prendre une décision directement sinon. De plus, si l’on avait un choix entre certaines évidences $\emph{X}_1, \emph{X}_2, \dotsc, \emph{X}_n$, l’on choisirait celle qui apporterait à $\emph{Eu}_\emph{X}$ une valeur maximale toujours en vérifiant que $\emph{Eu}_\emph{X} > 0$.
L’on constate que ces idées décrites ci-dessus sont utilisées assez souvent comme un fondement dans les autres articles y concernés lorsque l’on parle de la valeur d’information même si des formules ultérieures pour définir cet objet mathématique ne concordent pas parfois avec celles au-dessus.

\subsubsection{Approche Bayésienne}

Afin de s'avancer vers le problème d'élicitation il faut surtout décider comment pourrait-on représenter l'incertain de l'utilité de l'acteur particulier. Dans l’article \cite{chajewska_making_2000}, l’on fait une hypothèse assez cruciale : étant donné quel que soit une conséquence \emph{O} de décision quelconque \emph{D}, l’on considère que son utilité $\emph{u}_\emph{O}$ est plutôt une variable aléatoire qui suit telle ou telle loi de probabilité $\emph{P}(\emph{u}_\emph{O})$. Supposant plutôt empiriquement qu’une loi $\emph{P}(\textbf{u})$ (où $\textbf{u} = \{u_{O_1}, u_{O_2}, \dotsc, u_{O_m}\}$, et $\{O_1, O_2, \dotsc, O_m\}$ est un ensemble des conséquences possibles dans le domaine étudié) est une mixture de gaussiennes, peut-être tronquées, et disposant des statistiques assez exhaustives, l’on aura une capacité de reconstruire sa densité de probabilité pour ``utilisateur général''. Il faudrait quand même l’adapter à chaque fois pour chaque nouvel utilisateur car dans la vie réelle chacune et chacun ont leur propre utilité. C’est pour cette raison que l’on pose des questions de l’élicitation, réponses auxquelles mettent-à-jour une loi de probabilité de l’utilité. Bien que ces questions nous donnent des informations, généralement, utiles l’on ne doit pas les poser trop parce que l’utilisateur commence à un certain moment à se fatiguer de leur répondre. Par conséquent, il est nécessaire de déterminer un critère selon lequel l’on choisirait quelles questions exactement est-il mieux de poser. \cite{chajewska_making_2000} définit de manière assez similaire qu'au-dessus une dénotation de \emph{valeur de l’information} que l’on propose d’utiliser pour déterminer une question suivante à poser. Cependant, l’on remarque que cette approche est \emph{``myope''} car on ne considère qu’une valeur de l’information locale de chaque question. Une technique plus générale consisterait à observer toutes les combinaisons possibles que l’on pourrait construire à partir de questions prises en compte. Par contre, il est bien évident que pour la grande majorité de cas telle technique est intraitable à cause du nombre des combinaisons possibles qui augmente trop vite. C’est pourquoi, une approche proposée peut être une solution assez efficace afin de traiter ce problème-là.

Par ailleurs, l’article \cite{boutilier_pomdp_2002}  fournit une extension de \cite{chajewska_making_2000}  qui approfondit une idée de considérer une valeur d’utilité plutôt comme une variable aléatoire. En effet, l’on y propose de modéliser le problème de \emph{l’élicitation} par \emph{un processus de décision markovien partiellement observable} (POMDP pour \emph{Partially Observed Markov Decision Process} en anglais) qui est une généralisation d'\emph{un processus de décision markovien} (MDP pour \emph{Markov Decision Process} en anglais) où des chaînes des Markov simples sont remplacées par celles cachées. En revenant sur l’article mentionné, plus précisément, l’on y suppose que l’ensemble d’états du système modélisé est \emph{U}, l’ensemble de toutes les fonctions d’utilité possibles, alors que l’ensemble d’observations est tout simplement l’ensemble de toutes les densités de probabilité définies sur \emph{U}. De plus, le système est capable d’effectuer deux types d’actions : poser des questions et prendre une décision. Ensuite, l’on utilise des méthodes particulières développées pour POMDP afin de résoudre un problème de l’élicitation posé au début. Pour les raisons spatiales l’on omet des détails, pourtant, l’on remarque qu’une description ainsi que des explications sur l’approche au-dessus se trouvent dans l’article concerné.

Les techniques proposées dans \cite{chajewska_making_2000, boutilier_pomdp_2002} ont été développés dans \cite{braziunas_local_2005} pour le cas d’\emph{indépendance additive généralisée}. L’on y montre, plus spécifiquement, comment pourrait-on exploiter une élicitation myope utilisant une notation de \emph{valeur d’information} pour ce modèle-là. De plus, une structure graphique y développée de ce problème permet de nous concentrer sur l’élicitation locale ce qui nous donne alors un avantage principal du modèle d’\emph{indépendance} additive qui n’est pas en même temps aussi flexible que celui \emph{généralisé}. C’est-à-dire, l’on obtient un modèle assez générique qui autorise, cependant, l’élicitation locale. Une autre extension de \cite{chajewska_making_2000, boutilier_pomdp_2002} est l’article \cite{viappiani_optimal_2005} où l’on propose d’utiliser plutôt des \emph{ensembles optimaux de recommandations} pour choisir des questions à poser au lieu de déterminer l’une seule pas-à-pas. Cet article ci-dessus contient des aspects théorétiques de telle approche également que des résultats de simulations numériques.

\subsubsection{Approche ensembliste}

Une autre approche pour traiter l’élicitation est représentée dans \cite{iyengar_q-eval_2001}. Du fait, cette technique y développée a été proposée pour un problème légèrement différent : étant donné une séquence d’objets $\emph{o}_1, \emph{o}_2, \dotsc, \emph{o}_m$ chacun desquels disposent de \emph{n} attributs $\emph{x}_1, \emph{x}_2, \dotsc, \emph{x}_n$, comment pourrait-on déterminer ceux plus ou moins préférables pour un individu particulier ? De plus, comment pourrait-on définir l’utilité de chaque objet ? D’abord, supposons que l’utilité d’objet $\emph{o}$ = ($\emph{x}_1$, $\emph{x}_2$, $\dotsc$, $\emph{x}_n$) est représentée sous l’une des formes ci-dessous :
\[\emph{u}(\emph{o}) = \sum_{i = 1}^{\emph{n}} \emph{f}_i(\emph{x}_i) \times \emph{w}_i ,\]
\[\emph{u}(\emph{o}) = \prod_{i = 1}^{\emph{n}} (\emph{x}_i) ^ {\emph{w}_i} ,\]
où les fonctions $\emph{f}_1$, $\emph{f}_2$, $\dotsc$, $\emph{f}_n$ sont prédéfinies, en revanche, les poids $\emph{w}_1$, $\emph{w}_2$, $\dotsc$, $\emph{w}_ n$ ($\emph{w}_i \geq 0 $ $ \forall i = 1, 2, \dotsc, \emph{n}$) ne sont pas connus et l’on doit alors les trouver. Pour cela, l’on définit une région admissible \emph{P} dans laquelle $\emph{w}_1$, $\emph{w}_2$, $\dotsc$, $\emph{w}_ n$ peuvent se varier. Sans perte de généralité, soit \emph{P} défini initialement selon des contraintes ci-dessous
\[0 \leq \emph{w}_i \leq 1,  \forall i = 1, 2, \dotsc, \emph{n}-1, \]
\[\sum_{i = 1}^{\emph{n} - 1}\emph{w}_i \leq 1,\]
\[\emph{w}_n = 1 - \sum_{i = 1}^{\emph{n} - 1}\emph{w}_i. \]
Ayant l’ensemble \emph{P} bien déterminé l’on choisit son ``centre'' comme une solution initiale. Ensuite, pour obtenir ces poids de manière plus précise l’on propose itérativement à l’utilisateur de comparer certaines paires des objets mettant-à-jour à chaque fois la région \emph{P} et recalculant une solution $\emph{w}_1$, $\emph{w}_2$, $\dotsc$, $\emph{w}_ n$ en prenant un nouveau centre de \emph{P} actualisée. En outre, dès que l’on prend pour un résultat un point central de \emph{P} l’on cherche à minimiser un hyper volume de \emph{P} afin que les poids soient déterminés avec une erreur la plus petite que possible. C’est pour cette raison que choisissant une paire $(\emph{o}_i, \emph{o}_j)$ pour poser à l’individu afin de les comparer l’on prend celle qui nous permettrait de rétrécir le plus cette région \emph{P}. Finalement, l’on reboucle ce processus jusqu’à tant que le critère particulier de la terminaison soit vérifié (par exemple, l’on peut déclarer qu’un nombre des paires à poser ne peut pas être supérieur à quel que soit borne définie par avance pour que l’utilisateur ne commence pas à se fatiguer).
L’on remarque que ce problème peut être facilement reformulé comme un problème de l’élicitation de préférences en remplaçant des objets concernés par des conséquences possibles du système étudié et en supposant que chaque conséquence est définitivement déterminée par un vecteur concret des attributs au-dessus. Par ailleurs, une technique présentée exige certaines précisions sur des aspects qui ne sont peut-être pas assez claires :
\begin{itemize}
\item Comment pourra-t-on trouver un centre d’une région \emph{P} ?
\item De laquelle manière doit-on calculer un hyper volume que l’on coupera de \emph{P} posant telle ou telle paire $(\emph{o}_i, \emph{o}_j)$ ?
\end{itemize}
Des clarifications se trouvent quand même dans un article \cite{iyengar_q-eval_2001} auquel l’on réfère ainsi qu’un algorithme complet qui s’appelle \emph{Q-Eval}.

\subsection{Objectifs du projet}

Pour ce projet, on commence par une réalisation d'un logiciel qui permettra de résoudre des problèmes de \emph{Troubleshooting} différents à partir des leurs modèles donnés sous une forme de réseau Bayésien et en utilisant les algorithmes décrits dans les références ci-dessus, notamment \cite{Heckerman_1995, heckerman1994troubleshooting}. On cherchera ensuite des améliorations possibles pour ces algorithmes.

Mots-clés~: Troubleshooting, Value of Information.

\bibliographystyle{abbrv}
\bibliography{bibliographie}

\end{document}
